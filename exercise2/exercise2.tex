\documentclass{article}

\title{Cloud Computing - Exercise 2\\
Amazon EC2 - Scalable Web Service}
\author{Florian Feigenbutz - 346141\\
Tim Strehlow - 316594\\
Till Rohrmann - 343756}
\date{\today}

\begin{document}

\bibliographystyle{plain}

\maketitle

\setcounter{section}{0}
\section{Creating the AMI}
We enhanced our custom AMI from exercise one by executing the following steps:
\begin{enumerate}

	\item Use AMI created in exercise 1:
\begin{verbatim}
$ ec2-describe-images -o self

IMAGE	ami-8f80bbfb	228315521052/AMI for Assignment 1	
228315521052	available private		
x86_64		machine	aki-62695816instance-store	
paravirtual	xen
\end{verbatim}

	\item Start the instance to enhance the AMI
\begin{verbatim}
$ ec2-run-instances ami-8f80bbfb -k ubuntu-development 
--region eu-west-1 --availability-zone eu-west-1a -t m1.small

INSTANCE	i-cf7dd587	ami-8f80bbfb	
pending	ubuntu-development 0		m1.small	
2012-05-30T16:39:34+0000	eu-west-1a	aki-62695816
monitoring-disabled	instance-store		paravirtual
xen		sg-2c54a75b	default
\end{verbatim}

	\item Get instance address
\begin{verbatim}
$ ec2-describe-instances i-cf7dd587

INSTANCE	i-cf7dd587	ami-8f80bbfb	ec2-46-137-155-33.eu-west-1.compute.amazonaws.com 
ip-10-59-53-98.eu-west-1.compute.internal	running	ubuntu-development
0	m1.small	2012-05-30T16:39:34+0000	eu-west-1a	aki-62695816
monitoring-disabled	46.137.155.33	10.59.53.98	instance-store	paravirtual	xensg-2c54a75b	default
\end{verbatim}

	\item Connect to the machine
\begin{verbatim}
$ ssh -i [KEY] ubuntu@ec2-46-137-155-33.eu-west-1.compute.amazonaws.com
\end{verbatim}

	\item Install software
\begin{verbatim}
ubuntu@ip-10-59-53-98:~$ sudo apt-get install maven2 openjdk-7-jdk
\end{verbatim}

	\item Add user
\begin{verbatim}
ubuntu@ip-10-59-53-98:~$ sudo adduser primeserver
\end{verbatim}

	\item Copy files to primeserver user
\begin{verbatim}
$ scp -i [KEY] CC_SS12_Blatt2_additional_material.zip 
ubuntu@ec2-46-137-155-33.eu-west-1.compute.amazonaws.com
\end{verbatim}

	\item Unzip files
\begin{verbatim}
ubuntu@ip-10-59-53-98:~$ sudo -u primeserver 
  mv CC_SS12_Blatt2_additional_material.zip /home/primeserver
ubuntu@ip-10-59-53-98:~$ cd /home/primeserver
ubuntu@ip-10-59-53-98:~$ sudo -u primeserver 
  unzip CC_SS12_Blatt2_additional_material.zip
\end{verbatim}

	\item Register web service for auto start
\begin{verbatim}
ubuntu@ip-10-59-53-98:~$ sudo update-rc.d primeserver defaults
\end{verbatim}

	\item Change region in AMI manifest
\begin{verbatim}
ubuntu@ip-10-59-53-98:~$ sudo ec2-migrate-manifest -c /tmp/cert.pem 
-k /tmp/pk.pem -m /tmp/image.manifest.xml --region eu-west-1 
-a ACCESS_KEY -s SECRET_ACCESS_KEY
\end{verbatim}

	\item Bundle image
\begin{verbatim}
ubuntu@ip-10-59-53-98:~$ ec2-bundle-vol -k /mnt/keys/pk.pem 
  -u [USER_NUMBER] -c /mnt/keys/cert.pem
\end{verbatim}

	\item Upload AMI
\begin{verbatim}
ubuntu@ip-10-59-53-98:~$ ec2-upload-bundle --debug -b assignment2-ami 
  -m /tmp/image.manifest.xml --access-key [ACCESS_KEY] 
  --secret-key [SECRET_KEY] --url http://s3.amazonaws.com
\end{verbatim}

	\item Register AMI
\begin{verbatim}
ubuntu@ip-10-59-53-98:~$ ec2-register 
  assignment2-ami/image.manifest.xml -n "AMI for Assignment 2"

IMAGE	ami-2f86bc5b
\end{verbatim}
\end{enumerate}

\setcounter{section}{1}
\section{Setting up Auto Scaling with Alarms and Load Balancing}
Within the next steps we configured our web service to automatically scale up and down using 
a load balancer, alarms and auto scaling. Therefore we used the command line tools for 
AWS Elastic Load Balancing, AutoScaling and Cloud Watch.
In order to avoid passing the access and secret key for each command we adapted the file 
\verb|AWS_AUTO_SCALING_HOME|/credential-file-path.template and referenced it using the environment variable \verb|AWS_CREDENTIAL_FILE|.
\begin{enumerate}

	\item Set required environment variables for ELB and Cloud Watch
\begin{verbatim}
$ export AWS_ELB_URL=https://elasticloadbalancing.eu-west-1.amazonaws.com
$ export AWS_CLOUDWATCH_URL=https://monitoring.eu-west-1.amazonaws.com
\end{verbatim}

	\item Create ELB
\begin{verbatim}
$ elb-create-lb assignment2LB --availability-zones eu-west-1a 
  --listener "protocol=http, lb-port=9000, instance-port=9000"

DNS_NAME  assignment2LB-184147699.eu-west-1.elb.amazonaws.com
\end{verbatim}

	\item Create Launch Configuration
\begin{verbatim}
$ as-create-launch-config assignment2LC --image-id ami-2f86bc5b 
  --instance-type m1.small --key ubuntu-development

OK-Created launch config
\end{verbatim}

	\item Create Auto Scaling Group with at least 1 up to 6 Machines
\begin{verbatim}
$ as-create-auto-scaling-group assignment2ASG 
  --launch-configuration assignment2LC  --availability-zones eu-west-1a 
  --min-size 1 --max-size 6 --load-balancers assignment2LB

OK-Created AutoScalingGroup
\end{verbatim}

	\item Create Scaling Policy to Increase Number of Instances on High Load
\begin{verbatim}
$ as-put-scaling-policy assignment2UpScalePolicy 
  --auto-scaling-group assignment2ASG --adjustment=1 
  --type ChangeInCapacity --cooldown 300

arn:aws:autoscaling:eu-west-1:228315521052:scalingPolicy
  :28307e72-b97b-461c-b38c-331e8cd3bab4
  :autoScalingGroupName/assignment2ASG:policyName/assignment2UpScalePolicy
\end{verbatim}

	\item Create Watch Alarm for High CPU Load
\begin{verbatim}
$ mon-put-metric-alarm assignment2HighCPUAlarm 
  --comparison-operator GreaterThanOrEqualToThreshold --evaluation-periods 1 
  --metric-name CPUUtilization --namespace "AWS/EC2" --period 120 
  --statistic Average --threshold 90 
  --alarm-actions arn:aws:autoscaling:eu-west-1:228315521052:scalingPolicy
    :28307e72-b97b-461c-b38c-331e8cd3bab4:autoScalingGroupName/assignment2ASG:
    policyName/assignment2UpScalePolicy 
  --dimensions "AutoScalingGroupName=assignment2ASG"

OK-Created Alarm
\end{verbatim}

	\item Create Scaling Policy to Reduce Number of Instances on Low Load
\begin{verbatim}
$ as-put-scaling-policy assignment2DownScalePolicy 
  --auto-scaling-group assignment2ASG --adjustment=-1 
  --type ChangeInCapacity --cooldown 300

arn:aws:autoscaling:eu-west-1:228315521052:scalingPolicy
  :2d9dad1a-e996-4dab-8cb4-c63346155b66
  :autoScalingGroupName/assignment2ASG:policyName/assignment2DownScalePolicy
\end{verbatim}

	\item Create Watch Alarm for Low CPU Load
\begin{verbatim}
$ mon-put-metric-alarm assignment2LowCPUAlarm 
  --comparison-operator LessThanOrEqualToThreshold --evaluation-periods 1 
  --metric-name CPUUtilization --namespace "AWS/EC2" --period 120 
  --statistic Average --threshold 50 
  --alarm-actions arn:aws:autoscaling:eu-west-1:228315521052
    :scalingPolicy:2d9dad1a-e996-4dab-8cb4-c63346155b66
    :autoScalingGroupName/assignment2ASG:policyName/assignment2DownScalePolicy 
  --dimensions "AutoScalingGroupName=assignment2ASG"

OK-Created Alarm
\end{verbatim}
\end{enumerate}

\setcounter{section}{2}
\section{Test the Web Service Scaling using EC2 Instances as Clients}
Next we used another EC2 instance as a client to test the scalability of our infrastructure we set up within the last section.
\begin{enumerate}
	\item Start the Instance to Enhance the AMI
\begin{verbatim}
$ ec2-run-instances ami-2f86bc5b -k ubuntu-development 
  --region eu-west-1 --availability-zone eu-west-1a -t m1.small

RESERVATION	r-8a1546c3	228315521052	default
INSTANCE	i-0946ee41	ami-2f86bc5b	pending	
ubuntu-development	0	m1.small 2012-05-30T18:03:23+0000
eu-west-1a	aki-62695816 monitoring-disabled	
instance-store		paravirtual xen		sg-2c54a75b	default
\end{verbatim}

	\item Get Instance Address
\begin{verbatim}
$ ec2-describe-instances i-0946ee41

RESERVATION	r-8a1546c3	228315521052	default
INSTANCE	i-0946ee41	ami-2f86bc5b	ec2-176-34-95-58.eu-west-1.compute.amazonaws.com	
ip-10-248-81-158.eu-west-1.compute.internal	running	ubuntu-development	
0 m1.small	2012-05-30T18:03:23+0000	eu-west-1a	aki-62695816	
monitoring-disabled	176.34.95.58	10.248.81.158
instance-store		paravirtual	xensg-2c54a75b	default
\end{verbatim}

	\item Connect to the Machine
\begin{verbatim}
$ ssh -i [KEY] ubuntu@ec2-176-34-95-58.eu-west-1.compute.amazonaws.com
\end{verbatim}

	\item Start the Client
\begin{verbatim}
ubuntu@ip-10-248-81-158:~$ ~/assignment2/prime-client.sh 
  http://assignment2LB-184147699.eu-west-1.elb.amazonaws.com:9000/PrimeService 10
\end{verbatim}

	\item Periodically Check the Number of Assigned EC2 Instances
\begin{verbatim}
$ watch -n 30 elb-describe-instance-health assignment2LB

Every 30.0s: elb-describe-instance-health assignment2LB
Wed May 30 19:27:32 2012

INSTANCE_ID  i-49228a01  InService  N/A  N/A
INSTANCE_ID  i-5f208817  InService  N/A  N/A
INSTANCE_ID  i-3554fc7d  InService  N/A  N/A
INSTANCE_ID  i-932d85db  InService  N/A  N/A
INSTANCE_ID  i-532c841b  InService  N/A  N/A
INSTANCE_ID  i-4329810b  InService  N/A  N/A
\end{verbatim}

	\item Check Currently Active Instances
\begin{verbatim}
$ ec2-describe-instances -F instance-state-code=16

RESERVATION	r-e40e5dad	228315521052	default
INSTANCE	i-49228a01	ami-2f86bc5b	ec2-79-125-66-78.eu-west-1.compute.amazonaws.com 
...
\end{verbatim}
\end{enumerate}

\setcounter{section}{4}
\section{Conclusion}
With AWS Elastic Load Balancing, Auto Scaling Groups and Cloud Watch deploying a scalable and reliable infrastructure becomes quite comfortable.
Such a setup can be configured in many parameters to fit the project needs in terms of scalability and performance.
In our example we only used the CPU usage as a load indicator because our algorithm mostly uses CPU and therefore response times will 
nearly always equals the system machines' CPU load. Nevertheless one could imaging other scenarios in which it would be very helpful to track
more then a single indicator such e.g. memory or I/O usage or even rely on indicators that correlate with the end users' experience of the 
system performance.


\end{document}
